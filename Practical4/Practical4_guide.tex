\documentclass[pdflatex,a4paper]{article}

\usepackage{pgffor, ifthen}
\newcommand{\notes}[3][\empty]{%
    \noindent \vspace{10pt}\\
    \foreach \n in {1,...,#2}{%
        \ifthenelse{\equal{#1}{\empty}}
            {\rule{#3}{0.5pt}\\}
            {\rule{#3}{0.5pt}\vspace{#1}\\}
        }
}

\usepackage{graphicx}

\usepackage[margin=3cm]{geometry}

\title{Practical 4. Variables, Booleans, and loops}

\author{MI Stefan}

\date{IBI1, 2018/19}


\usepackage{url}
\usepackage{amsmath}

\begin{document}

\newcommand{\<}{\textless}
\renewcommand{\>}{\textgreater}


\maketitle

\section{Learning objectives}

\begin{itemize}
\item
Explain how variables work within a piece of code
\item
Explain the concept of loops
\item
Understand and use Booleans
\item
Plan a project using pseudocode
\item
Create and use variables in Python 
\item
Create and use loops and Booleans in Python
\end{itemize}

\section{Getting the folder for week 4}

\begin{itemize}
\item
First of all, you need to fetch the materials for week 4 from GitHub. Open up your repository \verb=IBI1_2018-19= in GitKraken.
\item
Add the original repository (not your fork) to the list of remotes. You can do this by going to ``Remote'' on the left side of your GitKraken window, clicking on the ``+'', and then selecting the GitHub Repo \verb=MelanieIStefan/IBI1_2018-19= from the pull-down menu, then clicking ``Add Remote''. 
\item
Once the remote has been added, go to the three dots next to ``Master'' under the MelanieIStefan remote repo. Click ``checkout MelanieIStefan/master''
\item
You now (confusingly) have two master branches, one from the course remote, and one that's your own. Merge them in such a way that you now have the new Practical 4 folder (plus content), but also your files in the Practical 3 folder haven't changed and are still yours.
\item
At the end of tutorial, don't forget to push your changes to your GitHub repo. 
\end{itemize}

\section{Working with Spyder}

\begin{itemize}
\item
You have installed Spyder as part of your anaconda-navigator installation. Start up Spyder.
\item
Explore the Spyder window. There is a console in the lower right part. You can use this to type in python commands directly and see the output, for instance \verb=1+1=
\item
The main window is for scripts. A python script is a collection of commands. You can write several lines of code and then run them all together by clicking ``Run''.  If there is a result, it will be shown in the console. Before you run the script for the first time, you will have to save it to a .py file.
\item
You can also run individual lines by selecting them and then clicking Run \(\rightarrow\) Run selection or current line
\item
Try it out: Type the following in a file called \verb=helloworld.py=
\begin{verbatim}
a = "hello "
b = "world!"
print(a+b)
\end{verbatim}
\item
Python files are not that different from text files. In particular, they can be put under version control using git and GitKraken. For more complex projects in particular, make sure you commit changes on a regular basis, so you can go back to earlier versions if you run into problems.
\end{itemize}


\section{Working with variables}

\begin{itemize}
\item
Start a new file called \verb=variables.py=
\end{itemize}


\subsection{Some simple math}

\begin{itemize}
\item
Think of a 3-digit number (e.g. 123). Store it in a variable called \verb=a=. 
\item
Create another variable called \verb=b= and assign to it a six-digit number, created by writing the digits of \verb=a= twice (in this example, 123123). 
\item
Can \verb=b= be divided by 7 (giving an integer)?  Use \verb=%= to check!  Create a new variable \verb=c=. Assign it to be \verb=b= divided by 7.
\item
Create a new variable \verb=d=. Assign it to \verb=c= divided by 11.
\item
Create a new variable called \verb=e= that is \verb=d= divided by 13.
\item
Compare \verb=e= to \verb=a=. Which of them is greater? Why?
\end{itemize}

\subsection{Booleans}

\begin{itemize}
\item
You learned in lecture that ``either X or Y'' is the same as ``(X and not Y) or (Y and not X)''. Make Boolean variables X and Y. Make a variable Z that encodes ``(X and not Y) or (Y and not X)'' and verify that it true if either X or Y (but not both) are true. Make a variable W that encodes Zhiwen's more elegant solution (``X !=Y'') and verify that W and Z are always the same, no matter the values of X and Y. 
\end{itemize}

\section{Collatz sequence}

\begin{itemize}
\item
Start a new file called \verb=collatz.py=
\item
For any positive integer \(n\), the next number in the Collatz sequence can be obtained by either dividing by 2 (if \(n\) is even) or multiplying by 3 and adding 1 (if \(n\) is odd). It seems that Collatz sequences always end with \(4-2-1-4-2-1-4-2-1-\dots\), no matter where they started. 
\item
Write a script that starts with a positive integer \verb=n= and computes and displays the Collatz sequence of \verb-n- and ends when you reach \(1\) for the first time. (Python may decide to display some figure(s) after the comma, e.g. ``25.0'' instead of ``25''. This is nothing to worry about at this stage.) Before you write the actual code, plan your project using pseudocode and comments (lines starting with \verb=#=). When you write the actual code, leave the pseudocode comments in.
\end{itemize}



\section{Mystery Code}
\begin{itemize}
\item
Look at file \verb=mysterycode.py=
\item
What does it do? Run it a few times and formulate a hypothesis. 
\item
Check whether your hypothesis is correct by going through the code line by line. Sometimes, it's helpful to do this using pen and paper, but it is up to you. If you find out what a particular line does, use comments (lines starting with \verb=#=) to note your thoughts. We have done this at the beginning of the document, e.g. in lines 4-6 to explain line 7. 
\item
Once you have confirmed what the code does, write a one-sentence description next to \verb=# Answer:= on line 2.
\end{itemize}




\section{Powers of 2}

\begin{itemize}
\item
Start a new file called \verb=powersof2.py=
\item
Every number can be written as a sum of powers of 2. For instance,

\(
2019 = 2^{10} + 2^{9} + 2^{8} + 2^{7} + 2^{6} + 2^{5} + 2^{1} + 2^{0}
\)
\item
Write some code that starts with some number \verb=x= (e.g. \verb_x=2019_) and computes the powers of 2 that make up \verb=x= (for simplicity, you can assume that \(x\) is no larger than \(8192=2^{13}\))  

The output should be a sentence on the screen that gives the binary composition of \verb=x=. For instance,

\begin{verbatim}
2019 is 2**10 + 2**9 + 2**8 + 2**7 + 2**6 + 2**5 + 2**1 + 2**0
\end{verbatim}
\item
Before you write the actual code, plan your project using pseudocode and comments (lines starting with \verb=#=). When you write the actual code, leave the pseudocode comments in.
\item
You may find the command \verb=str()= helpful: It converts a number into a string. For instance, \verb=str(5)= will give \verb="5"=
\end{itemize}




\section{For your portfolio}

The markers will look for and assess the following:

\begin{description}
\item[File variables.py] $\;$\\
\begin{itemize}
\item
The marker will look at your variables \(a\) and \(e\),  confirm that \(e\) has been created in the way that was specified in the instructions, and compare \(a\) to \(e\).
\item
The marker will test all possible values of \(X\) and \(Y\) and verify that \(W\) and \(Z\) behave as they should.
\end{itemize}

\item[File collatz.py] $\;$\\

\begin{itemize}
\item
The marker will set n to some number and check that the code runs, terminates, and that it does indeed display the Collatz sequence. 
\item
The marker will verify that you have used pseudocode to plan and comment your project. 
\end{itemize}


\item[File mysterycode.py] $\;$\\

\begin{itemize}
\item
The marker will look at the answer and check whether it is correct.
\end{itemize}



\item[File powersof2.py] $\;$\\

\begin{itemize}
\item
The marker will verify that you have used pseudocode to plan and comment your project
\item
The marker will test your script using the number \verb_x=1750_  
\item
Partial grades will be given if the project is incomplete.
\end{itemize}


\end{description}



You can add or edit things after the Practical session. We do not look at the commit date, we just want it all to be there!

\end{document}


